\graphicspath{{members/tf/figures/}}

\subsection{Pipeline A (sollte schon bei Saman anfgangen)}
\input{members/tf/authors}
\subsubsection{area transition}
In previous steps we used clustering to estimate the fraction if pixels in the image from above that dosplay leave surface. But this value is only a fraction and no area measure jet.\\
To translate that fraction into an area we have to find the total area that the camera observes first. To do that we take a look at the camera setup again.
   \begin{figure}[H]
       \centering
       \includegraphics[scale=0.6]{setupAbove.PNG}
       \caption{Setup of the camera above the plant. $\alpha$ marks the lense angle in which the camera captures the picture. $\alpha$ and $h_{Camera}$ are known.}
       \label{fig:setupAbove}
   \end{figure}{}
By only looking at half of the cone that the camera covers we can spot a triangle with a $90^{\circ}$ angle. Furthermore the height of the trianlge is $h_{Camera}$, the angle in the upper corner is $\frac{\alpha}{2}$ and the bottom side has a length of half the width of the observed area. The width of the observed area, calculated from the triangle needs to be squared, to get the area of the squared image:\\
$$A_{Camera} = (2\cdot h_{Camera}\cdot tan(\frac{\alpha}{2}))^2$$
$A_{Camera}$ only needs to be calculated once, as long as the camera parameters don't change. To calculate the area of the observed green, we simply multiply the captured area of the camera with the fraction $p_{Green}$ calculated by the clustering: 
$$A_{Observed} = p_{green}\cdot A_{Camera}$$
\textbf{Note:} This whole process works as long as the camera is set as supposed. A tilted camera would break the results since the used triangle would no longer have a $90^{\circ} angle. A tilted camera might also unintentionally capture adjacent plants.\\
\textbf{Model:} The observed plants are relatively small (max 40cm) in relation to the height of the camera (min 2m). Hence the plants can be considered to be flat on the ground, without changing the visible area. This assumption breaks when the plants grow much larger than expected. However this is pretty uncommon and can be therfore be neglected.
\subsubsection{tilt angle correction}
\subsection{Pipeline B}
\input{members/tf/authors}
\subsubsection{Height estimation setup}
\subsubsection{Estimating top of plant}
\subsubsection{Height estimation}
\subsection{Estimating LAI}
\subsection{Yield prediction}
\input{lorem}
